% Created 2015-01-01 Thu 08:53
\documentclass[11pt]{article}
\usepackage[utf8]{inputenc}
\usepackage[T1]{fontenc}
\usepackage{fixltx2e}
\usepackage{graphicx}
\usepackage{longtable}
\usepackage{float}
\usepackage{wrapfig}
\usepackage{rotating}
\usepackage[normalem]{ulem}
\usepackage{amsmath}
\usepackage{textcomp}
\usepackage{marvosym}
\usepackage{wasysym}
\usepackage{amssymb}
\usepackage{hyperref}
\tolerance=1000
\usepackage{listings}
\usepackage[usenames,dvipsnames]{color}
\usepackage{underscore}
\usepackage{parskip}
\usepackage{lmodern}
\usepackage{parskip}
\setlength{\parindent}{0pt}
\usepackage{underscore}
\textwidth 16cm
\oddsidemargin 0.5cm
\evensidemargin 0.5cm
\author{Heejung Shim}
\date{\today}
\title{"scripts used in Raj et al 2015 to run PIQ"}
\hypersetup{
  pdfkeywords={},
  pdfsubject={},
  pdfcreator={Emacs 24.3.1 (Org mode 8.2.10)}}
\begin{document}

\maketitle
\tableofcontents

To run \href{https://bitbucket.org/thashim/piq-single/}{PIQ}, we followed instructions from the authors of PIQ (see \href{https://bitbucket.org/thashim/piq-single/issue/5/running-piq-on-pre-selected-motif-sites}{here}, \href{https://bitbucket.org/thashim/piq-single/issue/6/running-piq-with-two-library-replicates}{here}, and \href{https://bitbucket.org/thashim/piq-single/issue/7/interpreting-piq-output}{here}). We have bam file for each replicate in
\lstset{breaklines=true,showspaces=false,showtabs=false,tabsize=2,basicstyle=\ttfamily,frame=single,keywordstyle=\color{Blue},stringstyle=\color{BrickRed},commentstyle=\color{ForestGreen},columns=fullflexible,language=bash,label= ,caption= ,numbers=none}
\begin{lstlisting}
~anilraj/histmod/cache/PIQdata/
\end{lstlisting}
and selected motifs sites for each TF as bed file in 
\lstset{breaklines=true,showspaces=false,showtabs=false,tabsize=2,basicstyle=\ttfamily,frame=single,keywordstyle=\color{Blue},stringstyle=\color{BrickRed},commentstyle=\color{ForestGreen},columns=fullflexible,language=bash,label= ,caption= ,numbers=none}
\begin{lstlisting}
~anilraj/histmod/cache/PIQ/<tfid>_Gm12878_sites.bed
\end{lstlisting}


First, we created directories:
\lstset{breaklines=true,showspaces=false,showtabs=false,tabsize=2,basicstyle=\ttfamily,frame=single,keywordstyle=\color{Blue},stringstyle=\color{BrickRed},commentstyle=\color{ForestGreen},columns=fullflexible,language=bash,label= ,caption= ,numbers=none}
\begin{lstlisting}
cd /mnt/lustre/home/shim/pbm_dnase_profile/analysis/piq/
mkdir motifsites
mkdir readfiles
mkdir tmp
mkdir res_pooled
mkdir res_Rep1
mkdir res_Rep2
\end{lstlisting}


We prepare R binary file from bam file, for each replicate:
\lstset{breaklines=true,showspaces=false,showtabs=false,tabsize=2,basicstyle=\ttfamily,frame=single,keywordstyle=\color{Blue},stringstyle=\color{BrickRed},commentstyle=\color{ForestGreen},columns=fullflexible,language=R,label= ,caption= ,numbers=none}
\begin{lstlisting}
/data/tools/R-3.1.1/bin/Rscript bam2rdata.r common.r /mnt/lustre/home/shim/pbm_dnase_profile/analysis/piq/readfiles/Rep1.RData ~anilraj/histmod/cache/PIQdata/Gm12878_Rep1.sort.bam
/data/tools/R-3.1.1/bin/Rscript bam2rdata.r common.r /mnt/lustre/home/shim/pbm_dnase_profile/analysis/piq/readfiles/Rep2.RData ~anilraj/histmod/cache/PIQdata/Gm12878_Rep2.sort.bam
\end{lstlisting}
and after pooling two replicates:
\lstset{breaklines=true,showspaces=false,showtabs=false,tabsize=2,basicstyle=\ttfamily,frame=single,keywordstyle=\color{Blue},stringstyle=\color{BrickRed},commentstyle=\color{ForestGreen},columns=fullflexible,language=R,label= ,caption= ,numbers=none}
\begin{lstlisting}
/data/tools/R-3.1.1/bin/Rscript bam2rdata.r common.r /mnt/lustre/home/shim/pbm_dnase_profile/analysis/piq/readfiles/pooled.RData ~anilraj/histmod/cache/PIQdata/Gm12878_pooled.sort.bam
\end{lstlisting}
and to run two replicates
\lstset{breaklines=true,showspaces=false,showtabs=false,tabsize=2,basicstyle=\ttfamily,frame=single,keywordstyle=\color{Blue},stringstyle=\color{BrickRed},commentstyle=\color{ForestGreen},columns=fullflexible,language=R,label= ,caption= ,numbers=none}
\begin{lstlisting}
/data/tools/R-3.1.1/bin/Rscript bam2rdata.r common.r /mnt/lustre/home/shim/pbm_dnase_profile
/analysis/piq/readfiles/multi.RData ~anilraj/histmod/cache/PIQdata/Gm12878_Rep1.sort.bam ~anilraj/histmod/cache/PIQdata/Gm12878_Rep2.sort.bam
\end{lstlisting}


We prepare motif sites input file for each TF (`\$\{tfid\}' indicates TF name): 
\lstset{breaklines=true,showspaces=false,showtabs=false,tabsize=2,basicstyle=\ttfamily,frame=single,keywordstyle=\color{Blue},stringstyle=\color{BrickRed},commentstyle=\color{ForestGreen},columns=fullflexible,language=R,label= ,caption= ,numbers=none}
\begin{lstlisting}
/data/tools/R-3.1.1/bin/Rscript bed2pwm.r common.r ~anilraj/histmod/cache/PIQ/S${tfid}_Gm12878_sites.bed S ${tfid} /mnt/lustre/home/shim/pbm_dnase_profile/analysis/piq/motifsites/
\end{lstlisting}
and run PIQ for each TF, for each replicate: 
\lstset{breaklines=true,showspaces=false,showtabs=false,tabsize=2,basicstyle=\ttfamily,frame=single,keywordstyle=\color{Blue},stringstyle=\color{BrickRed},commentstyle=\color{ForestGreen},columns=fullflexible,language=R,label= ,caption= ,numbers=none}
\begin{lstlisting}
/data/tools/R-3.1.1/bin/Rscript pertf.r common.r /mnt/lustre/home/shim/pbm_dnase_profile/analysis/piq/motifsites/ /mnt/lustre/home/shim/pbm_dnase_profile/analysis/piq/tmp/ /mnt/lustre/home/shim/pbm_dnase_profile/analysis/piq/res_Rep1/ /mnt/lustre/home/shim/pbm_dnase_profile/analysis/piq/readfiles/Rep1.RData ${tfid}

/data/tools/R-3.1.1/bin/Rscript pertf.r common.r /mnt/lustre/home/shim/pbm_dnase_profile/analysis/piq/motifsites/ /mnt/lustre/home/shim/pbm_dnase_profile/analysis/piq/tmp/ /mnt/lustre/home/shim/pbm_dnase_profile/analysis/piq/res_Rep2/ /mnt/lustre/home/shim/pbm_dnase_profile/analysis/piq/readfiles/Rep2.RData ${tfid}
\end{lstlisting}
and after pooling two replicates:
\lstset{breaklines=true,showspaces=false,showtabs=false,tabsize=2,basicstyle=\ttfamily,frame=single,keywordstyle=\color{Blue},stringstyle=\color{BrickRed},commentstyle=\color{ForestGreen},columns=fullflexible,language=R,label= ,caption= ,numbers=none}
\begin{lstlisting}
/data/tools/R-3.1.1/bin/Rscript pertf.r common.r /mnt/lustre/home/shim/pbm_dnase_profile/analysis/piq/motifsites/ /mnt/lustre/home/shim/pbm_dnase_profile/analysis/piq/tmp/ /mnt/lustre/home/shim/pbm_dnase_profile/analysis/piq/res_pooled/ /mnt/lustre/home/shim/pbm_dnase_profile/analysis/piq/readfiles/pooled.RData ${tfid}
\end{lstlisting}
and run two replicates
\lstset{breaklines=true,showspaces=false,showtabs=false,tabsize=2,basicstyle=\ttfamily,frame=single,keywordstyle=\color{Blue},stringstyle=\color{BrickRed},commentstyle=\color{ForestGreen},columns=fullflexible,language=R,label= ,caption= ,numbers=none}
\begin{lstlisting}
/data/tools/R-3.1.1/bin/Rscript pertf.r common.r /mnt/lustre/home/shim/pbm_dnase_profile/analysis/piq/motifsites/ /mnt/lustre/home/shim/pbm_dnase_profile/analysis/piq/tmp/ /mnt/lustre/home/shim/pbm_dnase_profile/analysis/piq/res_multi/ /mnt/lustre/home/shim/pbm_dnase_profile/analysis/piq/readfiles/multi.RData ${tfid}
\end{lstlisting}
% Emacs 24.3.1 (Org mode 8.2.10)
\end{document}
